\documentclass[11pt]{article}

% Multi language
\usepackage[english]{babel}
% Unicode package
\usepackage[utf8]{inputenc}
% Citation package
\usepackage[authoryear]{natbib}
% Set margins and line space
\usepackage{geometry}
\geometry{verbose, tmargin=2cm, bmargin=2cm, lmargin=2cm, rmargin=2cm}
\usepackage{setspace}
\onehalfspacing

% Math Packages
\usepackage{amsmath}
\numberwithin{equation}{section}
\usepackage{amsfonts}
\usepackage{amsthm}
\usepackage{amssymb}
\usepackage{mathrsfs}
\usepackage{bbm}
\usepackage{amstext}

% For tables
\usepackage{multirow}
\usepackage{babel}
\usepackage{array}
\usepackage{rotating}
\usepackage{longtable}
\usepackage{float}
\usepackage{booktabs}
\usepackage{lscape}

%%%%%%%%%%%%%%%%%%%%%%%%%%%%%% LyX specific LaTeX commands.
%% Because html converters don't know tabularnewline
\providecommand{\tabularnewline}{\\}

% For figures
\usepackage{caption}
\usepackage{subcaption}

% For algorithms
\usepackage[ruled]{algorithm2e}

% For code
\usepackage{xcolor}
\usepackage{listings}
\usepackage{alltt}

\definecolor{codegreen}{rgb}{0,0.6,0}
\definecolor{codegray}{rgb}{0.5,0.5,0.5}
\definecolor{codepurple}{rgb}{0.58,0,0.82}
\definecolor{backcolour}{rgb}{0.95,0.95,0.92}

\lstdefinestyle{mystyle}{
    backgroundcolor=\color{backcolour},   
    commentstyle=\color{codegreen},
    keywordstyle=\color{magenta},
    numberstyle=\tiny\color{codegray},
    stringstyle=\color{codepurple},
    basicstyle=\ttfamily\footnotesize,
    breakatwhitespace=false,         
    breaklines=true,    captionpos=b,                    
    keepspaces=true, numbers=left,                    
    numbersep=5pt, showspaces=false,                
    showstringspaces=false, showtabs=false, tabsize=2}
\lstset{style=mystyle}


% Graphics
\usepackage{color}
\usepackage{graphicx}

% maxwidth is the original width if it is less than linewidth
% otherwise use linewidth (to make sure the graphics do not exceed the margin)
\makeatletter
\def\maxwidth{ %
  \ifdim\Gin@nat@width>\linewidth
    \linewidth
  \else
    \Gin@nat@width
  \fi
}
\makeatother

\definecolor{fgcolor}{rgb}{0.345, 0.345, 0.345}
\newcommand{\hlnum}[1]{\textcolor[rgb]{0.686,0.059,0.569}{#1}}%
\newcommand{\hlstr}[1]{\textcolor[rgb]{0.192,0.494,0.8}{#1}}%
\newcommand{\hlcom}[1]{\textcolor[rgb]{0.678,0.584,0.686}{\textit{#1}}}%
\newcommand{\hlopt}[1]{\textcolor[rgb]{0,0,0}{#1}}%
\newcommand{\hlstd}[1]{\textcolor[rgb]{0.345,0.345,0.345}{#1}}%
\newcommand{\hlkwa}[1]{\textcolor[rgb]{0.161,0.373,0.58}{\textbf{#1}}}%
\newcommand{\hlkwb}[1]{\textcolor[rgb]{0.69,0.353,0.396}{#1}}%
\newcommand{\hlkwc}[1]{\textcolor[rgb]{0.333,0.667,0.333}{#1}}%
\newcommand{\hlkwd}[1]{\textcolor[rgb]{0.737,0.353,0.396}{\textbf{#1}}}%
\let\hlipl\hlkwb

\usepackage{framed}
\makeatletter
\newenvironment{kframe}{%
 \def\at@end@of@kframe{}%
 \ifinner\ifhmode%
  \def\at@end@of@kframe{\end{minipage}}%
  \begin{minipage}{\columnwidth}%
 \fi\fi%
 \def\FrameCommand##1{\hskip\@totalleftmargin \hskip-\fboxsep
 \colorbox{shadecolor}{##1}\hskip-\fboxsep
     % There is no \\@totalrightmargin, so:
     \hskip-\linewidth \hskip-\@totalleftmargin \hskip\columnwidth}%
 \MakeFramed {\advance\hsize-\width
   \@totalleftmargin\z@ \linewidth\hsize
   \@setminipage}}%
 {\par\unskip\endMakeFramed%
 \at@end@of@kframe}
\makeatother

\definecolor{shadecolor}{rgb}{.97, .97, .97}
\definecolor{messagecolor}{rgb}{0, 0, 0}
\definecolor{warningcolor}{rgb}{1, 0, 1}
\definecolor{errorcolor}{rgb}{1, 0, 0}
\newenvironment{knitrout}{}{} % an empty environment to be redefined in TeX


% URL Setup
\usepackage{url}
% \usepackage[hidelinks]{hyperref}
\usepackage{hyperref}
\hypersetup{
    colorlinks=true,
    linkcolor=[rgb]{0.6000000, 0.1058824, 0.1176471}, % usc red.
    filecolor=[rgb]{0.6000000, 0.1058824, 0.1176471}, 
    urlcolor=[rgb]{0.6000000, 0.1058824, 0.1176471},
    citecolor=[rgb]{0.6000000, 0.1058824, 0.1176471},
    linkbordercolor=[rgb]{1.0000000, 0.7803922, 0.1725490}, % usc yellow
    citebordercolor=[rgb]{1.0000000, 0.7803922, 0.1725490}, 
    filebordercolor=[rgb]{1.0000000, 0.7803922, 0.1725490}, 
    urlbordercolor=[rgb]{1.0000000, 0.7803922, 0.1725490} 
    }
\urlstyle{same}

% Sections numbered within parts
\usepackage{chngcntr}
\counterwithin*{section}{part}

\renewcommand{\harvardurl}{URL: \url}  

% Utilities
\usepackage{todonotes} 
\newcommand{\CommentZhan}[1]{\todo[inline, color=blue!40]{\textbf{Comment (Zhan): }#1}} 
\newcommand{\CommentJing}[1]{\todo[inline, color=pink!40]{\textbf{Comment (Jing): }#1}} 
\usepackage{rotfloat}
\usepackage{pdfpages}
\usepackage{indentfirst}
\usepackage{enumitem}

% Define theorem environment
\newtheorem{theorem}{Theorem}
\newtheorem{remark}{Remark}
\newtheorem{assumption}{Assumption}
\newtheorem{lemma}{Lemma}
\newtheorem{notation}{Notation}
\newtheorem{example}{Example}
\newtheorem{definition}{Definition}
\newtheorem{proposition}{Proposition}
\newtheorem{corollary}{Corollary}

% Define math operators
\DeclareMathOperator{\argmax}{argmax}
\DeclareMathOperator{\argmin}{argmin}
\DeclareMathOperator{\tr}{tr}
\DeclareMathOperator{\vect}{vec}
\DeclareMathOperator{\Rank}{Rank}
\DeclareMathOperator{\plim}{plim}

% Define short cut
\newcommand{\Op}{\mathcal{O}_p}
\newcommand{\sq}[1]{\textquoteleft #1\textquoteright}
\newcommand{\dq}[1]{\textquotedblleft #1\textquotedblright}
\newcommand{\red}{\color{red}}
\newcommand{\E}{\mathbb{E}}

% Set counter
\setcounter{tocdepth}{2}

\AtBeginDocument{\hypersetup{pdfborder={0 0 1}}}


% -----------------------------------------------------------------------
% --------------------------- Document starts ---------------------------
% -----------------------------------------------------------------------

\title{Adversarial Estimation of Heterogeneous Treatment Effects}
\date{}

\begin{document}
\maketitle

% \tableofcontents
% \abstract{
%     The abstract. 
%     \\
%     \\
%     \\
%     Key words: keyword 1, keyword 2, keyword 3 \\
%     JEL code: code 1, code 2, code 3
% }
% \newpage

\section{Heterogenous treatment effect}

Consider a heterogeneous treatment effect model,
\begin{equation}
    Y_{i}  = \tau _{i}  d_{i} + f\left( \bm{X} _i \right) + u_i,\;  i \in \mathcal{N},
\end{equation}
where the treatment \(d_i\) is assigned in a random experiment, and \(\bm{X} _{i} \in \mathbb{R}^{p} \) is the observed characteristics. Both treated and control units are drawn from the same super population.
Denote 
\(\mathcal{T} =\left\{ i \in \mathcal{N} : d _{i} = 1 \right\} \) and
\(\mathcal{C} = \mathcal{N} \setminus \mathcal{T} \) 
as the sets of treated and control units, respectively, and correspondingly \(N_{1} = \vert \mathcal{T}  \vert \) and \(N_{0} = \vert \mathcal{C}  \vert \).

Suppose \(\ \left\{ \tau _{i}  \right\} _{i \in \mathcal{T} }\)  is known, define
\begin{equation}
    \tilde{Y}_{i}  = \begin{cases}
        Y_{i}  & \text{ if }  d_{i} = 0, \\
        Y_i -\tau _{i} & \text{ if }  d _{i} = 1,
    \end{cases} \quad i \in \mathcal{N},
\end{equation}
then
\[
    \tilde{Y}_{i} = f\left( \bm{X} _{i}  \right) + U_i,\, \forall i\in \mathcal{N}, 
\]
i.e. \(S_{\mathcal{T} } = \left\{ \tilde{Y} _{i} , \bm{X} _{i}  \right\}_{i\in \mathcal{T} } \) and \(S_{\mathcal{\mathcal{C} } } = \left\{ \tilde{Y} _{i} , \bm{X} _{i}  \right\}_{i\in \mathcal{\mathcal{C} } } \) follow the same data generating process. In this case, one cannot distinguish \(S_{\mathcal{T} }\) and \(S_{\mathcal{C} }\). 

In practice, the heterogeneous treatment effects \(\tau _{i} \) are unknown parameter of interests. In \citet{wager/athey:2018:rf_hte}, \(\tau _{i} \) is modeled,
\[
    \tau \left( \bm{x} \right)  = \E \left( \tau _{i} \vert \bm{X}_{i} = \bm{x} \right).
\]
\begin{remark}
    Let \(\tau _{i}  = \tau \left( \bm{X}_{i}  \right) + v _{i} \),
    \begin{equation*}
        Y _{i}  = \tau \left( \bm{X} _{i}  \right) d_{i} + f \left( \bm{X} \right) + u_{i} + v_{i} d _{i}.  
    \end{equation*}
    The question is whether the difference \(v_{i} d_{i} \) is learnable.
\end{remark}

Following the intuition from the case where \(\left\{ \tau _{i}  \right\}_{i \in \mathcal{T} }\) is known, we propose to adopt the generative adversarial network (GAN) framework \citep{goodfellow:2014:GAN,kaji/manresa/pouliot:2022:adversarial_est} to estimate \(\left\{ \tau _{i}  \right\} _{i\in \mathcal{T} }\). Consider a minimax game between two components, a generator \(G\)  and a discriminator \(D\) , which can be modeled as deep neural networks. The estimation problem is defined as
\begin{equation}
    \min_{G\in \mathcal{G} } \max_{D \in \mathcal{D} } 
    \frac{1}{N_{1}} \sum_{i \in \mathcal{T} } 
        \log D\left( \tilde{Y} _{i} \left( G\left( \bm{X} _{i}  \right)  \right), \bm{X} _{i}  \right) + 
    \frac{1}{N_{0}} \sum_{i\in \mathcal{C} } \log \left( 
        1 - D\left( Y _{i} , \bm{X} _{i}  \right)
     \right),
\end{equation}
in which the inner maximization problem looks for a discriminator distinguishing treated and control samples whereas the outer minimization trains a generator that adversarially generates treatment effects \(\tau _{i} \). 

\section{Literature}

\citet{yoon/jordan/vanderschaar:2018:ganite} use GAN to estimate the individual treatment effect by generating the counterfactual outcomes. \citet{liang2021well,liang:2021:gan_distribution,liang:2018:gan_density} set up the theoretical ground. \citet{liang:2021:gan_distribution} focus on the moments of the target distribution, whereas we are interested in distinguishing two distributions.

\newpage
\bibliographystyle{chicago}
\bibliography{ref}

% \newpage
% \appendix

% \begin{center}
% {\huge Appendix}
% \end{center}

% \setcounter{table}{0} \renewcommand{\thetable}{A.\arabic{table}} %
% \setcounter{section}{0} \renewcommand{\thesection}{A.\arabic{section}} %
% \setcounter{figure}{0} \renewcommand{\thefigure}{A.\arabic{figure}}




% \section{Appendix section}

% \newpage
% \begin{center}
%     {\huge Online Supplements}
% \end{center}d

% \setcounter{table}{0} \renewcommand{\thetable}{S.\arabic{table}} %
% \setcounter{section}{0} \renewcommand{\thesection}{S.\arabic{section}} %
% \setcounter{figure}{0} \renewcommand{\thefigure}{S.\arabic{figure}}


\end{document}